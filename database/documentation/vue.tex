\chapter{Les vues - fonctionnement général}\label{vue}

\chapter{Génération du menu}
Pour les pages web, le menu est généré de manière dynamique :
\begin{itemize}
\item lors du premier appel à l'application ;
\item après toute opération de connexion ou de déconnexion.
\end{itemize}

Le menu est stocké en variable de session, pour accélérer l'affichage.

Il est structuré sous la forme d'une liste non ordonnée (balises ul et li), et contient les classes utilisées par bootstrap pour son affichage.

\section{Fichier de description}

Le menu est généré à partir du fichier \textbf{param/menu.xml}. La branche principale s'appelle \textit{<menu>}. Voici un exemple d'entrée, qui correspond au menu d'administration :

\begin{lstlisting}
<item module="administration" value="4" title="5" droits="admin">
	<item module="loginList" droits="admin" title="3" value="2"/>
	<item module="appliList" droits="admin" value="appliliste" title="applilistetitle"/>
	<item module="aclloginList" droits="admin" value="aclloginliste" title="aclloginlistetitle"/>
	<item module="groupList" droits="admin" value="groupliste" title="grouplistetitle"/>
	<item module="phpinfo" droits="admin" value="phpinfo" title="phpinfotitle"/>
</item>
\end{lstlisting}

Les entrées du menu sont déclarées dans des balises \textbf{item}. Voici les attributs utilisables :

\begin{longtable}{|p{2.5cm}|c|p{9cm}|}
\hline
\textbf{Attribut} & \textbf{Requis} & \textbf{Signification} \\
\hline
\endhead
module & X & Nom du module à exécuter, tel que décrit dans le fichier actions.xml (\textit{cf.} \ref{actions} \textit{\nameref{actions}}, page \pageref{actions})\\
 \hline
droits & & Droit nécessaire pour afficher l'entrée du menu. Il est possible d'indiquer plusieurs droits, en les séparant par une virgule\\
 \hline
loginrequis & & Si vaut 1, l'entrée ne sera affichée que si l'utilisateur est connecté \\
 \hline
onlynoconnect & & Si vaut 1, l'entrée ne sera affichée que si l'utilisateur n'est pas connecté\\
 \hline
value & X & nom de la sous-variable du tableau \$LANG["menu"], qui contient le libellé à afficher (\textit{cf.} \ref{langue} \textit{\nameref{langue}}, page \pageref{langue})\\
 \hline
title & X & nom de la sous-variable du tableau  \$LANG["menu"], qui contient le libellé à afficher au survol de la souris (attribut HTML \textit{title})  (\textit{cf.} \ref{langue} \textit{\nameref{langue}}, page \pageref{langue})\\
 \hline

\caption{Liste des attributs utilisables pour décrire les entrées du menu}
\end{longtable}

Une entrée \textit{item} peut contenir d'autres entrées \textit{item}, ce qui permet de décrire les menus en cascade. Actuellement, le menu n'a été testé qu'avec 2 niveaux (menu principal horizontal, et menus verticaux associés).

L'ordre d'affichage est celui décrit dans le fichier xml.

\section{Génération en mode développement}

Si la variable \textit{APPLI\_modeDeveloppement} est positionnée à \textit{true}, le menu est généré à chaque appel.

\chapter{Gestion des langues}\label{langue}



\chapter{La vue Smarty}\label{smarty}

\chapter{Les autres vues}
\section{La vue Ajax}
\section{La vue CSV}
\section{La vue PDF}
