\chapter{Les vues}\label{vue}

D'une manière générale, toute action demandée se termine par l'exécution d'une vue : envoi d'une page HTML -- le cas le plus fréquent --, envoi d'un fichier au format JSON pour les requêtes de lecture AJAX, envoi de fichiers dans des formats variés : fichiers PDF, CSV, des images...

Chaque type d'envoi nécessite une vue différente. Les actions demandées (les modules appelés) décrivent quelle vue doit être utilisée (\textit{cf.} \ref{actions} \textit{\nameref{actions}}, page \pageref{actions}). Toutefois, certains modules ne sont pas associés à des vues : ce sont ceux qui vont écrire des informations dans la base de données, et qui enchaîneront systématiquement sur un autre module qui, lui, déclenchera un affichage.

Les vues sont toutes héritées d'une classe de base, \textbf{Vue}, qui ne devrait pas être instanciée. Cette classe contient les fonctions génériques suivantes :

\begin{longtable}{|p{5cm}|p{8cm}|}
\hline
\textbf{fonction} & \textbf{Objectif} \\
\hline
\endhead

set(\$value, \$variable = "") & stocke une valeur dans la vue. Le nom de la variable n'est fourni que pour certains types de vues. Si une seule valeur est stockée sans indiquer de nom, elle peut être utilisée telle qu'elle \\
\hline
send(\$param = "") & déclenche l'envoi du contenu. Elle doit être systématiquement réécrite (vide par défaut). \\
\hline
encodehtml(\$data) & encode la variable fournie avant un envoi vers le navigateur. C'est une fonction récursive capable de traiter les tableaux imbriqués \\
\hline

\caption{Fonctions déclarées dans la classe non instanciable Vue}
\end{longtable}

\section{La vue Smarty}

Il s'agit de la vue la plus utilisée dans le Framework. Elle permet de générer les pages web.

\subsection{Fonctions disponibles}

\subsection{Organisation de l'écran}


\section{La vue Ajax}
Nom de la classe : \textbf{VueAjaxJson}.

Elle encode le tableau fourni par la fonction \textit{set()} au format Json, et transmet la chaîne générée au navigateur, après avoir nettoyé le cache.

\section{La vue CSV}

Nom de la classe : \textbf{VueCsv}.

Fonctions disponibles : 
\begin{longtable}{|p{5cm}|p{8cm}|}
\hline
\textbf{fonction} & \textbf{Objectif} \\
\hline
\endhead
setFilename(\$filename) & indique le nom à utiliser pour générer le fichier\\
\hline
send(\$param = "") & Déclenche l'envoi du tableau vers le navigateur, au format CSV. \textit{param} peut contenir le nom du fichier souhaité. S'il est vide, le nom du fichier transmis par la fonction précédente est utilisé. Sinon, un nom de fichier, contenant la date, est généré.\\
\hline
\caption{Fonctions déclarées dans la classe VueCsv}
\end{longtable}

La fonction set() doit être utilisée pour indiquer le tableau à transformer en CSV. La classe va générer automatiquement une ligne d'entête à partir du nom des colonnes de la première ligne.

En l'état actuel, il n'est pas possible de définir des options particulières pour la génération du fichier CSV.

\section{La vue binaire}

Nom de la classe : \textbf{VueBinaire}.

Cette vue est utilisée pour envoyer des données sous forme binaire au navigateur (images, par exemple). Les données doivent avoir été auparavant générées dans un fichier du serveur web : c'est le contenu du fichier qui est transmis.

Fonctions disponibles : 
\begin{longtable}{|p{5cm}|p{8cm}|}
\hline
\textbf{fonction} & \textbf{Objectif} \\
\hline
\endhead
setParam(array \$param) & transmet un tableau contenant l'ensemble des paramètres à utiliser pour générer le fichier. Les paramètres sont les suivants : \\
& \textit{filename} : nom du fichier tel qu'il apparaîtra dans le navigateur \\
& \textit{disposition} : \textit{attachment} (fichier joint) ou \textit{inline} (affichage direct dans le navigateur) \\
& \textit{tmp\_name} : nom du fichier dans le serveur \\
& \textit{content\_type} : type mime. S'il n'est pas indiqué, le programme essaiera de le déterminer à partir du contenu du fichier \\
\hline
send() & Envoie le fichier au navigateur, en fonction des paramètres indiqués \\
\hline
\caption{Fonctions déclarées dans la classe VueBinaire}
\end{longtable}

\section{La vue PDF}

Nom de la classe : \textbf{VuePdf}.

Il s'agit d'une variante de la vue précédente. Elle accepte non pas le nom d'un fichier, mais la référence correspondant à une fonction \textit{fopen()} ou équivalente. Cette approche est nécessaire si le fichier PDF à envoyer à été stocké dans une base de données ouverte avec PDO.

Fonctions disponibles :
\begin{longtable}{|p{5cm}|p{8cm}|}
\hline
\textbf{fonction} & \textbf{Objectif} \\
\hline
\endhead

setFileReference(\$ref)& indique la référence du fichier à traiter (résultat de fopen() ou d'une lecture PDO)\\
\hline
setFilename(\$filename)& Nom du fichier tel qu'il sera transmis au navigateur. S'il n'est pas précisé, il sera généré (en cas d'attachement) \\
\hline
setDisposition(\$disp = "attachment")& Indique la manière d'envoyer le fichier au navigateur. Valeurs acceptées : \textit{attachment} ou \textit{inline}\\
\hline
send() & Envoie le fichier au navigateur, en fonction des paramètres indiqués \\
\hline
\caption{Fonctions déclarées dans la classe VuePdf}
\end{longtable}

La classe peut générer des exceptions en cas de problème.

\chapter{Génération du menu}
Pour les pages web, le menu est généré de manière dynamique :
\begin{itemize}
\item lors du premier appel à l'application ;
\item après toute opération de connexion ou de déconnexion.
\end{itemize}

Le menu est stocké en variable de session, pour accélérer l'affichage.

Il est structuré sous la forme d'une liste non ordonnée (balises ul et li), et contient les classes utilisées par bootstrap pour son affichage.

\section{Fichier de description}

Le menu est généré à partir du fichier \textbf{param/menu.xml}. La branche principale s'appelle \textit{<menu>}. Voici un exemple d'entrée, qui correspond au menu d'administration :

\begin{lstlisting}
<item module="administration" value="4" title="5" droits="admin">
	<item module="loginList" droits="admin" title="3" value="2"/>
	<item module="appliList" droits="admin" value="appliliste" title="applilistetitle"/>
	<item module="aclloginList" droits="admin" value="aclloginliste" title="aclloginlistetitle"/>
	<item module="groupList" droits="admin" value="groupliste" title="grouplistetitle"/>
	<item module="phpinfo" droits="admin" value="phpinfo" title="phpinfotitle"/>
</item>
\end{lstlisting}

Les entrées du menu sont déclarées dans des balises \textbf{item}. Voici les attributs utilisables :

\begin{longtable}{|p{2.5cm}|c|p{9cm}|}
\hline
\textbf{Attribut} & \textbf{Requis} & \textbf{Signification} \\
\hline
\endhead
module & X & Nom du module à exécuter, tel que décrit dans le fichier actions.xml (\textit{cf.} \ref{actions} \textit{\nameref{actions}}, page \pageref{actions})\\
 \hline
droits & & Droit nécessaire pour afficher l'entrée du menu. Il est possible d'indiquer plusieurs droits, en les séparant par une virgule\\
 \hline
loginrequis & & Si vaut 1, l'entrée ne sera affichée que si l'utilisateur est connecté \\
 \hline
onlynoconnect & & Si vaut 1, l'entrée ne sera affichée que si l'utilisateur n'est pas connecté\\
 \hline
value & X & nom de la sous-variable du tableau \$LANG["menu"], qui contient le libellé à afficher (\textit{cf.} \ref{langue} \textit{\nameref{langue}}, page \pageref{langue})\\
 \hline
title & X & nom de la sous-variable du tableau  \$LANG["menu"], qui contient le libellé à afficher au survol de la souris (attribut HTML \textit{title})  (\textit{cf.} \ref{langue} \textit{\nameref{langue}}, page \pageref{langue})\\
 \hline

\caption{Liste des attributs utilisables pour décrire les entrées du menu}
\end{longtable}

Une entrée \textit{item} peut contenir d'autres entrées \textit{item}, ce qui permet de décrire les menus en cascade. Actuellement, le menu n'a été testé qu'avec 2 niveaux (menu principal horizontal, et menus verticaux associés).

L'ordre d'affichage est celui décrit dans le fichier xml.

\section{Génération en mode développement}

Si la variable \textit{APPLI\_modeDeveloppement} est positionnée à \textit{true}, le menu est généré à chaque appel.

\chapter{Gestion des langues}\label{langue}

Le framework a été conçu pour supporter plusieurs langues européennes. Pour cela, les libellés à afficher peuvent être stockés dans un tableau, \$LANG, qui sera chargé en fonction de la langue demandée.

Les fichiers contenant les libellés (les entrées du tableau) sont placés dans le dossier \textit{locales}.

La variable \$language, dans le fichier \textit{param.default.inc.php} (\textit{cf.} \ref{param} \textit{\nameref{param}}, page \pageref{param}) contient le nom de la langue par défaut. Le fichier correspondant au code (fr.php) est systématiquement chargé. Si une autre langue est demandée, le second fichier sera alors lu, et les nouveaux libellés seront traités en remplacement : cela permet de conserver les libellés d'origine s'ils n'ont pas été traduits.

Le tableau \$LANG est organisé en plusieurs sous-tableaux :
\begin{itemize}
\item menu : contient tous les libellés utilisés dans les menus ;
\item message : libellés généraux utilisés dans l'ensemble de l'application ;
\item login : libellés utilisés dans le module de gestion des droits et des utilisateurs ;
\item ObjetBDDError : messages correspondants aux anomalies détectées par la classe ObjetBDD (\textit{cf.} \ref{objetbdd} \textit{\nameref{objetbdd}}, page \pageref{objetbdd}) ;
\item les autres entrées sont libres pour l'application.
\end{itemize}

Les libellés peuvent inclure des balises HTML, qui seront envoyées telles qu'elles au navigateur. Cela permet d'insérer, par exemple, un retour à la ligne (<br>).

La variable \$LANG est transmise systématiquement à la vue Smarty.

\section{Formatage des dates}
Le fichier de langue commence par la définition du format de date, qui est ensuite transmis à ObjetBDD pour que la classe en tienne compte lors du formatage des informations. 

\chapter{Compléments sur Smarty}\label{smarty}

