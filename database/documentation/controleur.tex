\chapter{Fonctionnement général}
\begin{figure}[h]
\includegraphics[width=1.2\linewidth]{dessin/synopsis}
\caption{Synopsis général de fonctionnement du contrôleur}
\end{figure}



\section{Synopsis}

L'appel de toute page dans l'application passe nécessairement par l'ensemble de ces étapes :
\begin{itemize}
\item vérification que l'encodage des caractères transmis respecte bien l'encodage utf-8
\item lecture des paramètres
\item chargement des classes génériques utilisées systématiquement
\item démarrage de la session, et ajout de contrôles (durée de la session ouverte...)
\item lecture des paramètres en sur-écrasement, ce qui permet des implémentations multiples avec le même code
\item initialisation de l'identification
\item contrôles de cohérence IP (vérification que, pour une même session, l'adresse IP ne change pas)
\item lancement des connexions aux bases de données (par défaut, deux connexions : une pour la base des droits, l'autre pour les données applicatives)
\item décodage des variables HTML encodées (protection contre les attaques de type XSS)
\item traitement du module demandé :
\begin{itemize}
\item initialisation, le cas échéant, de la vue associée 
\item vérification de l'identification, ou déclenchement des procédures d'identification
\item vérification des droits nécessaires pour accéder au module
\item vérification, le cas échéant, de la cinématique : les opérations de modification ne devraient être possibles que si l'opération précédente correspond à l'affichage du formulaire de saisie
\item exécution du module
\item analyse du code de retour du module, et enchaînement le cas échéant sur un autre module
\end{itemize}
\item déclenchement de la vue 
\item enregistrement, le cas échéant, des messages destinés à SYSLOG (messages systèmes)
\end{itemize}

\section{Organisation des dossiers}
Les fichiers sont organisés selon cette arborescence :
\begin{itemize}
\item \textbf{database} : dossier de travail contenant la description de la base de données, la documentation pour les développeurs, les scripts. Le dossier doit être supprimé lors de la mise en production
\item \textbf{display} : le seul dossier accessible. Il contient tous les fichiers nécessaires pour gérer l'affichage :
\begin{itemize}
\item \textbf{CSS} : les feuilles de style
\item \textbf{images} : les icônes et images utilisées dans l'affichage des pages
\item \textbf{javascript} : l'ensemble des librairies Javascript utilisées
\item \textbf{templates} : les modèles de documents utilisés par Smarty (cf. \ref{smarty}, \textit{\nameref{smarty}}, page \pageref{smarty})
\item \textbf{templates\_c} : dossier utilisé par Smarty pour compiler les templates. Ce dossier doit être accessible en écriture par le serveur Web
\end{itemize}
\item \textbf{doc} : ancien dossier, contenant un mécanisme de gestion de la documentation en ligne. N'est plus utilisé actuellement, mais pourrait être employé le cas échéant
\item \textbf{framework} : le code de base du framework. Il comprend :
\begin{itemize}
\item \textbf{droits} : dossier permettant de gérer les droits 
\item \textbf{identification} : gestion de la connexion des utilisateurs
\item \textbf{import/import.class.php} : classe créée il y a quelques années pour gérer les imports (obsolète en grande partie)
\item \textbf{ldap/ldap.class.php} : connexion à un annuaire LDAP et récupération d'informations
\item \textbf{navigation} : programmes utilisés pour générer le menu et décoder les actions demandées à partir du fichier XML les contenant
\item \textbf{translateId/translateId.class.php} : classe permettant de transcoder les identifiants des enregistrements de la base de données, pour éviter les attaques par forçage de clé
\item de nombreux fichiers utilisés par le framework, dont le contrôleur (controller.php), des fonctions génériques (fonctions.php)...
\item \textbf{vue.class.php} : les classes utilisées pour les vues (cf. \ref{vue} \textit{\nameref{vue}}, page \pageref{vue})
\end{itemize}
\item \textbf{install} : contient des scripts d'installation de la base de données (normalement à déplacer dans \textit{database}), et le fichier \textbf{readme.txt}, décrivant les dernières nouveautés
\item \textbf{locales} : dossier contenant les fichiers de langue (fr.php et en.php)
\item \textbf{modules} : dossier contenant le code spécifique de l'application. Il est organisé ainsi :
\begin{itemize}
\item \textbf{classes} : les classes nécessaires pour l'application
\item \textbf{example} : des exemples de codage
\item les autres dossiers sont libres et contiennent les modules de l'application
\item \textbf{beforeDisplay.php} : fichier appelé systématiquement avant l'affichage des pages HTML
\item \textbf{beforesession.inc.php} : fichier appelé systématiquement avant le démarrage de la session. Il permet de déclarer les librairies qui sont nécessaires pour instancier des classes stockées en variables de session
\item \textbf{common.inc.php} : fichier appelé systématiquement avant le traitement des modules
\item \textbf{fonctions.php} : fonctions déclarées par le programmeur et disponibles dans toute l'application 
\item \textbf{postLogin.php} : script exécuté uniquement quand un utilisateur s'est identifié
\end{itemize}
\item \textbf{param} : dossier contenant les paramètres de l'application :
\begin{itemize}
\item \textbf{actions.xml} : fichier contenant la description de l'ensemble des modules utilisables, avec les droits associés et le type de vue à utiliser
\item \textbf{menu.xml} : description du menu qui sera généré
\item \textbf{param.default.inc.php} : les paramètres par défaut
\item \textbf{param.inc.php} : paramètres en écrasement, spécifiques de l'implémentation. Ce fichier n'est jamais livré lors des mises à jour, pour éviter la suppression des paramètres de base de données, par exemple
\item \textbf{param.inc.php.dist} : fichier d'exemple de \textit{param.inc.php}, à renommer et à mettre à jour lors de l'installation d'une nouvelle implémentation
\end{itemize}
\item \textbf{plugins} : dossier contenant les bibliothèques tierces, comme Smarty, ObjetBDD (maintenant intégré au framework)...
\item \textbf{temp} : dossier de stockage temporaire, qui doit être accessible en écriture au serveur web. Les fichiers présents dans celui-ci ont une durée de vie de 24 heures (suppression lors de la connexion d'un utilisateur)
\item \textbf{test} : dossier utilisé pour réaliser certains tests. Doit être systématiquement supprimé lors de la mise en production
\end{itemize}

Seuls le fichier index.php, à la racine, les dossiers display et test sont accessibles directement. Les autres dossiers sont protégés par des fichiers .htaccess.

\section{Paramètres}

Les paramètres utilisés dans l'application sont gérés avec 3 fichiers différents :
\begin{itemize}
\item \textbf{param/param.default.inc.php} : contient l'ensemble des paramètres utilisés ;
\item \textbf{param/param.inc.php} : contient ceux issus du fichier précédent, qui sont adaptés à l'implémentation ;
\item \textbf{param.ini} : fichier contenant les paramètres spécifiques du nom DNS de l'application (par exemple, schéma particulier associé au nom du site). Pour plus d'informations sur ce point, consultez le chapitre \ref{dnsmultiple} \textit{\nameref{dnsmultiple}}, page \pageref{dnsmultiple}.
\end{itemize}

Voici la description de l'ensemble des paramètres :

\subsection{Paramètres généraux}
% \usepackage{array} is required
\begin{longtable}{|p{5cm}|p{8cm}|}
\hline
\textbf{Variable} & \textbf{Signification} \\
\hline
\endhead
APPLI\_version & Numéro de version de l'application \\ 
\hline 
APPLI\_versiondate & Date de la version \\ 
\hline 
language & Langue par défaut \\
\hline
DEFAULT\_formatdate & Format par défaut d'affichage des dates\\
\hline
navigationxml & nom du fichier XML contenant la description des modules exécutables\\
\hline
APPLI\_session\_ttl & durée de la session, en secondes\\
\hline
APPLI\_cookie\_ttl & durée de vie par défaut des cookies, en secondes\\
\hline
APPLI\_path\_stockage\_session & obsolète\\
\hline
LOG\_duree & Durée de conservation des traces des actions réalisées, en jours\\
\hline
APPLI\_mail & Adresse pour déclarer les incidents (mail ou non)\\
\hline
APPLI\_titre & Nom de l'application qui sera affiché (cas où le code est utilisé par plusieurs entrées différentes) \\
\hline
APPLI\_code & Code interne de l'application. Utilisé dans certains cas\\
\hline
APPLI\_fds & Feuille de style utilisée par défaut (obsolète)\\
\hline
APPLI\_address & Adresse DNS de l'application. Utilisée en cas d'identification CAS (adresse de retour)\\
\hline
APPLI\_modeDeveloppement & si à true, certaines opérations sont réalisées dans un contexte de développement (affichage de messages, recalcul systématique du menu...)\\
\hline
APPLI\_notSSL & utilisé en développement, si l'application ne fonctionne pas en mode SSL (déconseillé) \\
\hline
APPLI\_utf8 & systématiquement à true (plus de support des autres encodages)\\
\hline
APPLI\_menufile & nom du fichier XML contenant la description du menu\\
\hline
APPLI\_temp & nom du dossier utilisé pour stocker les fichiers temporaires\\
\hline
APPLI\_moduleDroitKO & nom du module appelé en cas de refus d'accès pour un problème de droits \\
\hline
APPLI\_moduleErrorBefore & nom du module appelé en cas de problème lié à la cinématique de l'application\\
\hline
APPLI\_moduleNoLogin & nom du module appelé en cas d'échec d'identification \\
\hline
paramIniFile & nom du fichier contenant les paramètres spécifiques liés au DNS utilisé (\textit{cf.} \ref{dnsmultiple} \textit{\nameref{dnsmultiple}}, page \pageref{dnsmultiple}) \\
\hline
SMARTY\_param & Paramètres utilisés par le moteur de templates SMARTY\\
\hline
SMARTY\_variables & variables systématiquement transmises à SMARTY et utilisées lors de l'affichage général\\
\hline
ERROR\_display & Affiche les erreurs à l'écran (mode développement)\\
\hline
OBJETBDD\_debugmode & 0 : pas d'affichage de message d'erreur, 1, affichage des messages d'erreur, 2 : affichage de toutes les commandes SQL générées par ObjetBDD \\
\hline
ADODB\_debugmode & obsolète \\
\hline
\caption{Variables générales de l'application}
\end{longtable} 

\subsection{Identification}
\begin{longtable}{|p{5cm}|p{8cm}|}
\hline
\textbf{Variable} & \textbf{Signification} \\
\hline
\endhead
ident\_type & Type d'identification supporté. L'application peut gérer \textbf{BDD} (uniquement en base de données),\textbf{LDAP} (uniquement à partir d'un annuaire LDAP) \textbf{LDAP-BDD} (d'abord identification en annuaire LDAP, puis en base de données), et \textbf{CAS} (serveur d'identification \textit{Common Access Service})\\
\hline
CAS\_plugin & Nom du plugin utilisé pour une connexion CAS \\
\hline
CAS\_address & Adresse du serveur CAS\\
\hline
CAS\_port & Systématiquement 443 (connexion chiffrée)\\
\hline
LDAP & tableau contenant tous les paramètres nécessaires pour une identification LDAP \\
\hline
privateKey & clé privée utilisée pour générer les jetons d'identification \\
\hline
pubKey & clé publique utilisée pour générer les jetons d'identification \\
\hline
tokenIdentityValidity & durée de validité, en secondes, des jetons d'identification\\
\hline
\caption{Variables utilisées pour paramétrer l'identification}
\end{longtable}

Voici le contenu des variables du tableau LDAP : 
\begin{longtable}{|p{5cm}|p{8cm}|}
\hline
\textbf{Variable} & \textbf{Signification} \\
\hline
\endhead
address &  adresse de l'annuaire\\
\hline
port & 389 en mode non chiffré, 636 en mode chiffré\\
\hline
rdn & compte de connexion, si nécessaire \\
\hline
basedn & base de recherche des utilisateurs\\
\hline
user\_attrib & nom du champ contenant le login à tester\\
\hline
v3 & toujours à \textit{true}\\
\hline
tls & \textit{true} en mode chiffré\\
\hline
groupSupport & \textbf{true} si l'application recherche les groupes d'appartenance du login dans l'annuaire\\
\hline
groupAttrib & Nom de l'attribut contenant la liste des groupes d'appartenance\\
\hline
commonNameAttrib & Nom de l'attribut contenant le nom de l'utilisateur\\
\hline
mailAttrib & Nom de l'attribut contenant l'adresse mail de l'utilisateur\\
\hline
attributgroupname & Attribut contenant le nom du groupe lors de la recherche des groupes (cn par défaut)\\
\hline
attributloginname & attribut contenant les membres d'un groupe\\
\hline
basedngroup & base de recherche des groupes \\
\hline
\caption{Variables utilisées pour paramétrer l'accès à l'annuaire LDAP}
\end{longtable}

\subsection{Connexions aux bases de données}

Deux connexions sont systématiquement implémentées : l'une à la base de données contenant la gestion des droits, et l'autre à celle contenant les données propres à l'application.
\begin{longtable}{|p{5cm}|p{8cm}|}
\hline
\textbf{Variable} & \textbf{Signification} \\
\hline
\endhead
BDD\_login & compte de connexion à la base de données \\
\hline
BDD\_passwd & mot de passe associé\\
\hline
BDD\_dsn & adresse de la base de données sous forme normalisée\\
\hline
BDD\_schema & schéma utilisé (plusieurs schémas peuvent être décrits, en les séparant par une virgule - fonctionnement propre à Postgresql)\\
\hline
GACL\_dblogin & compte de connexion à la base de données des droits\\
\hline
GACL\_dbpasswd & mot de passe associé\\
\hline
GACL\_dsn & adresse normalisée \\
\hline
GACL\_schema & schéma utilisé\\
\hline
GACL\_aco & nom du code de l'application utilisé dans la gestion des droits (\textit{cf.} \ref{droits} \textit{\nameref{droits}}, page \pageref{droits} )\\
\hline
 & \\
\hline
 & \\
\hline
 & \\
\hline

\caption{Variables utilisées pour paramétrer les connexions}
\end{longtable}


\chapter{Décrire les actions}

\chapter{Identification des utilisateurs et gestion des droits}\label{droits}